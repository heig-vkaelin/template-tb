\section{Conclusion technique}

% Etat final du projet, ce qui est fait / pas fait

% - Évaluation des choix faits, choses apprises, adaptation des solutions (critique positive/négative de l’implémentation)

\subsection{État final du projet}

Pour commencer, les diverses fonctionnalités listée dans le cahier des charges initial \ref{sec:cdc} sont répertoriées à nouveau ci-dessous. Pour chacune d'entre elles, il est indiqué si elle a été implémentée ou non.

\subsubsection{Fonctionnalités \guillemotleft required\guillemotright}

\begin{todolist}
  \item[\done] L'utilisateur arrive sur une page affichant la toile virtuelle au complet.
  \item[\done] La toile s'actualise avec les modifications réalisées par les autres utilisateurs.
  \item[\done] L'utilisateur peut zoomer ou dé-zoomer sur la toile afin de voir les moindres détails.
  \item[\done] L'utilisateur voit facilement le nombre de pixels qu'il a actuellement la possibilité de placer sur la toile.
  \item[\done] L'utilisateur peut sélectionner un pixel, choisir une couleur parmi plusieurs proposées et le colorier.
  \item[\done] Les dimensions de la toile doivent être configurables par un administrateur.
  \item[\done] L'utilisateur dispose d'un moyen de recharger ses pixels à placer une fois ceux-ci écoulés. Il peut par exemple recevoir un nombre de pixels définis après un temps d'attente également choisi.
\end{todolist}

\subsubsection{Fonctionnalités \guillemotleft essential\guillemotright}

\begin{todolist}
  \item[\done] Il doit être possible de passer la toile dans un mode “lecture uniquement” par un administrateur pour tous les utilisateurs.
  \item[\done] Créer des tests de montée en charge de l'application afin d'assurer un bon fonctionnement lors des festivals notamment.
  \item[\done] Afin d'éviter que les utilisateurs puissent recharger la page pour recevoir à nouveau des pixels, trouver un moyen de les identifier.
  \item[\done] Les coordonnées du pixel sélectionné par l'utilisateur lui sont affichées ainsi que son niveau de zoom.
  \item Créer des tests unitaires ainsi que des tests d'intégration pour s'assurer de la qualité de l'application.
  \item[\done] Un administrateur peut choisir une zone à censurer (recouvrir/suppression de pixels d'une couleur) à partir d'un call API en cas de comportement non désiré.
\end{todolist}

\subsubsection{Fonctionnalités \guillemotleft nice-to-have\guillemotright}

\begin{todolist}
  \item Ajouter une interface utilisateur permettant aux administrateurs de censurer plus facilement une zone de la toile via le dashboard.
  \item Une fois ses pixels épuisés, l'utilisateur peut les recharger via un moyen physique.
  \item[\done] Faire en sorte que les couleurs disponibles aux utilisateurs soient facilement customizables par l'administrateur (ex: via le dashboard).
  \item Les administrateurs peuvent interdire la pose de pixels à des utilisateurs ou des régions spécifiques.
  \item[\done] Les administrateurs peuvent changer la taille de la toile dynamiquement (ex : via le dashboard).
  \item[\done] Afficher à chaque utilisateur le nombre de festivaliers actuellement connectés sur la toile.
  \item[\done] Un mode “affichage”, permettant d'afficher régulièrement un QRCode pour rejoindre la toile virtuelle. En plus de ce QRCode, ajouter des éléments sollicitant l'interaction de l'utilisateur à la manière d'un économiseur d'écran (ex: des pixels qui s'animent de façon indépendante). Ce mode s'élèverait automatiquement dès qu'un utilisateur est actif sur l'application.
  \item Ajouter la possibilité de sauvegarder le dessin.
  \item Intégrer la notion de CRDTs (Conflict-free Replicated Data Type), une structure de données permettant d'éviter les conflits notamment dans les systèmes distribués collaboratifs multi utilisateurs.
\end{todolist}

\subsubsection{Synthèse}

Comme la liste ci-dessus l'indique, quasiment toutes les fonctionnalité "required" et "essential" ont été implémentées. En effet, seul les tests unitaires ainsi que les tests d'intégration n'ont pas été réalisés lors de ce travail de Bachelor. L'utilité de ce type de tests est indéniable mais il a été décidé de ne pas les implémenter pour le moment afin de se concentrer sur les fonctionnalités principales de l'application. De plus, les différentes pipelines de déploiement continu permettent de s'assurer que l'application restait dans un état sans erreur après chaque modification.

Concernant les fonctionnalités "nice-to-have", environ la moitié ont été implémentées. La majorité du travail restant concerne le dashboard d'administration. Ces fonctionnalités sont pour la plupart implémentées mais dans une version initiale sous la forme d'appels HTTP. Il est clair qu'avoir un dashboard dédié dans le futur facilitera la tâche des administrateurs.

Pour finir, le concept intéressant des CRDTs n'a pas été abordé dans ce travail de Bachelor. Ce point sera abordé plus en profondeur dans le chapitre \ref{perspectives}.

\subsubsection{Fonctionnalités supplémentaires}

% Choses pas prévues faites en plus
% - nombre users
% - package code commun

TODO

\section{Conclusion personnelle}

% - Comment le projet a été vécu

TODO






% \section{Défis rencontrés et futurs}

% Les principales difficultés rencontrées lors de cette première partie de projet concernent le frontend. Elles sont les suivantes :

% \begin{itemize}
%   \item Créer un PinchZoom qui prend en compte les différents types d'appareils, que ce soit un ordinateur, un mobile ou le pavé tactile.
%   \item Synchroniser le state côté frontend avant d'avoir utilisé une librairie de gestion de state global (Zustand).
% \end{itemize}

% Heureusement, aucun gros blocage n'a été rencontré et les difficultés ont pu être surmontées en y consacrant le temps nécessaire. Pour la suite du projet, le point proposant le plus gros défi est le fait de réussir à tester correctement la montée en charge de l'application. En effet, il ne faut pas que l'ordinateur lançant les tests soit l'élément limitant la montée en charge à la place de l'application. Ce point sera donc à prendre en compte et des solutions comme utiliser plusieurs ordinateurs/serveurs pourront être envisagées.