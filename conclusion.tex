Ce rapport intermédiaire signe la fin de la troisième Milestone et l'avancée du Travail de Bachelor est assez conforme au planning prévu. Le seul élément qui n'a pas été complété est le démarrage d'utilisation des tests de montées en charges qui sera reporté à la prochaine Milestone. Cependant, la première version développée de \gls{beeplace} pour le Baleinev Festival se trouve être plus qu'une simple base de travail pour la suite du projet. En effet, quasiment la totalité des fonctionnalités \guillemotleft{} required \guillemotright{} du cahier des charges ont été implémentées. Cela permettra d'avoir le temps de se concentrer sur les différents problèmes évoqués dans la dernière section \ref{poc-ameliorations} ainsi que sur les nombreuses fonctionnalités permettant d'améliorer l'expérience utilisateur.

\vfil
\hspace{8cm}\makeatletter\@author\makeatother\par
\hspace{8cm}\begin{minipage}{5cm}
  %%if
  % Place pour signature numérique
  % TODO: rajouter signature
  % \printsignature
  %%fi
\end{minipage}