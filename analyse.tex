\section{Architecture}

\fig[H, width=1\textwidth]{Diagramme de séquence de la pose de pixels}{sequence-overview.svg}

\subsection{Séparation Client / Serveur}

Tout d'abord, le choix de découper le projet en deux applications bien distinctes s'est fait rapidement. En effet, il aurait été possible de faire une application monolithique, comme il était plus commun de le faire à l'époque avec des frameworks comme Ruby on Rails ou Laravel. De nos jours, il est plus courant de bien séparer les différentes couches d'une application et d'en faire des applications indépendantes. Cela octroie plusieurs avantages:

\begin{itemize}
  \item Meilleure scalabilité, en pouvant notamment multiplier les instances de chaque application séparément.
  \item Développement plus facile entre plusieurs équipes de développeurs, sans devoir tous modifier le même code.
  \item Possibilité d'utiliser le même serveur pour plusieurs clients différents, par exemple pour un site web ainsi qu'une application mobile.
\end{itemize}

Cependant, il faut également prendre en compte que cette approche demande souvent plus de travail qu'une application unique. Par exemple, pour gérer l'état actuel de l'application, les deux partis doivent garder une synchronisation entre eux. Il faut donc bien faire attention à ce que les changements soient effectués des deux côtés.

\section{Choix des technologies}

Avant d'énumérer les technologies choisies, il faut noter que le temps pour réaliser le travail de Bachelor étant assez restreint, il est préférable de choisir une majorité de technologies déjà connues avec lesquelles le temps de prise en main est donc réduit.

\subsection{TypeScript}

TypeScript \cite{typescript} est un langage de programmation créé par Microsoft qui est un sur-ensemble de JavaScript. Il est donc possible de compiler du code TypeScript en JavaScript, ce qui permet de l'utiliser dans n'importe quel environnement. TypeScript peut donc être utilisé dans l'application entière, que ce soit du côté de l'application frontend dans le navigateur ou du côté backend sur le serveur grâce à Node.js \cite{nodejs}, qui permet de faire tourner du JavaScript en dehors du navigateur.

TypeScript ajoute de nombreuses fonctionnalités à JavaScript, comme la possibilité de créer des types, des interfaces, des énumérations et bien d'autres. Les avantages à utiliser TypeScript sont donc nombreux:

\begin{itemize}
  \item Meilleure complétion dans les IDE, comme Visual Studio Code.
  \item Gestion automatique des imports dans les IDE.
  \item Evite de découvrir de nombreuses erreurs au runtime, car elles sont détectées au moment de la compilation.
  \item Refactoriser le code de façon plus sure en voyant directement les erreurs de types dans l'IDE.
  \item Lecture et compréhension du code plus facile, car les types sont explicitement définis.
\end{itemize}

De plus, les deux applications existantes, Pimp My Wall et le Media Player, sont déjà écrites en TypeScript. Il est donc agréable d'avoir une certaine cohérence et uniformité dans les technologies utilisées.

\subsection{Frontend}

- React (vs Next.js) vs Vue.js
- Tailwind (PAS BOOTSTRAP)
- Voir autre libs potentiellement ?

\subsection{Backend}

Comme expliqué précédemment, le choix s'est tourné vers Node.js \cite{nodejs} afin d'utiliser un seul et même langage pour l'entièreté de l'application: TypeScript. Node.js permet d'utiliser les connaissances acquises en JavaScript côté client également pour le côté serveur. De plus, Node.js est un environnement très populaire, ce qui permet, en combinaison avec \Gls{npm}, de trouver des librairies pour presque toutes les problématiques possibles. De plus, les performances de Node.js sont très bonnes et il est possible de réaliser un scaling horizontal facilement en multipliant le nombre d'instances de l'application.

Pour finir, grâce à l'utilisation de TypeScript dans les deux applications, il est possible de partager du code, notamment les types entre les deux applications en créant un package. Cela permet d'éviter la duplication de code ainsi qu'être sûr de la cohérence des données.

Il existe de nombreux frameworks dans l'écosystème Node.js, voici les principaux encore actuellement maintenus:

\begin{table}[h]
  \begin{center}
    \caption{Frameworks Backend Node.js}
    \begin{tabular}{c|l|r}
      Framework  & Nombre de stars & Type      \\ \hline
      Express.js & 60.7k           & Librairie \\
      AdonisJS   & 13.8k           & Framework \\
      Fastify    & 27.1k           & Librairie \\
      NestJS     & 56.2k           & Framework \\
    \end{tabular}
  \end{center}
\end{table}

Bien que le plus populaire, Express n'est pas à proprement parler un framework, mais plutôt une librairie. Il est donc nécessaire d'utiliser une multitude d'autres librairies pour gérer les routes, les middlewares, la validation et autre. Pour des applications web complexes, cela revient à essayer d'assembler un puzzle en essayant de faire fonctionner toutes les briques ensemble. C'est pourquoi il est souvent préférable d'utiliser un framework plus dogmatique et plus complet. Fastify, bien que plus rapide et plus moderne que son concurrent Express, se trouve dans la même catégorie des frameworks très minimalistes. Il reste donc AdonisJS et NestJS. Le premier promet d'inclure tout ce dont nous avons besoin afin d'être le plus productif possible. AdonisJS est souvent comparé au framework PHP Laravel sur cet aspect-là. Le soucis principal d'AdonisJS est sa popularité. Malheureusement, il s'agit du framework le moins connu de la liste et cela se fait ressentir dans l'écosystème. Il est plus compliqué de trouver des librairies pour des cas d'utilisation spécifiques. La dernière option est NestJS, qui est un framework de plus en plus populaire et très complet. Son architecture fait qu'il est très facilement extensible, notamment grâce à l'utilisation de modules et d'injection de dépendances. De plus, son implémentation des WebSockets est très complète et mise en avant contrairement à AdonisJS, ce qui est un point très important pour la réalisation de cette application qui contient majoritairement des communications temps réel. Le choix s'est donc porté sur NestJS.

\subsection{Communication temps réel}

Pour la communication en temps réel sur le web, nous avons principalement trois options. Les Server-Sent Events (SSE), les WebSockets et WebRTC.

\subsubsection{Server-Sent Events (SSE)}
Utiliser des Server-Sent Events (SSE) est une option possible et facile à mettre en place. Cependant, cette technologie est assez limitée et ne permet pas de faire de la communication bidirectionnelle. En effet, pour réaliser une communication du client au serveur il faudrait utiliser en addition des appels HTTP classiques. Il est donc préférable d'utiliser une autre technologie afin d'uniformiser les communications. De plus, SSE supporte uniquement les communications textuels, il n'est pas possible de transférer des données binaires, ce qui peut être limitant.

\subsubsection{WebRTC}
WebRTC permet de réaliser une communication bidirectionnelle entre deux clients sans passer par un serveur une fois la connexion établie. Ce concept, appelé pair-à pair ou peer to peer en anglais, est intéressant pour des projets dans lesquels les utilisateurs créent des salons privés entre amis afin de ne pas surcharger le serveur. Cependant dans le cas de BeePlace, il est nécessaire de passer par le serveur afin de pouvoir synchroniser l'état de la toile entre tous les utilisateurs. WebRTC n'est donc pas une option viable pour ce projet.

\subsubsection{WebSockets}
Pour finir, les WebSockets mettent à disposition tout ce qui est nécessaire pour réaliser une communication bidirectionnelle entre le client et le serveur. De plus, de nombreuses librairies sont disponibles dans l'écosystème JavaScript et TypeScript afin de rendre son utilisation le plus simple possible. La plus connue est Socket.io et les deux fonctionnalités les plus intéressantes qu'elle nous offre sont les suivantes:

\begin{itemize}
  \item Auto reconnexion: Socket.io gère automatiquement les reconnexions en cas de perte de connexion.
  \item Création de salons: Il est possible de créer des salons afin de limiter les communications à un groupe d'utilisateurs. Cette fonctionnalité peut être utile si nous souhaitons par exemple créer plusieurs toiles simultanément sur l'application.
  \item Intégration avec Redis: Il est possible d'utiliser Redis afin de stocker les salons et les utilisateurs connectés. Cela permet de faciliter le scaling horizontal en multipliant le nombre d'instances de l'application.
\end{itemize}

Le choix s'est donc porté sur les WebSockets et plus spécifiquement sur la librairie Socket.io pour la communication temps réel.

\subsection{Stockage}

\subsubsection{Bases de données}

L'idée de séparer le stockage de l'état de la toile entre plusieurs méthodes de stockage est venue assez naturellement. En effet, la toile actuelle doit être accessible très rapidement afin d'éviter un temps de chargement initial trop important à l'utilisateur lorsque celui-ci arrive sur la page. De plus, il peut être intéressant de stocker des métadonnées en plus des pixels actuelles de la toile. Ces métadonnées se composent de l'historique des pixels avec la date et l'utilisateur qui l'a placé afin de pouvoir retracer l'évolution de la toile ainsi qu'en faire diverses analyses. Pour conclure, il est nécessaire d'avoir deux types de stockage pour l'application: un très rapide pour l'état actuel de la toile et un second plus durable pour l'historique et diverses métadonnées.

\subsubsection{Stockage de l'état actuel de la toile}

Afin de garantir un accès rapide à l'état actuel de la toile, un cache doit être mis en place. Les deux options les plus fréquentes sont Redis ou simplement une structure de données dans la mémoire de l'application. La deuxième option est plus simple à mettre en place car elle ne demande aucun service tiers mais le cache devrait être recréé depuis le stockage durable à chaque redémarrage de l'application. Ce qui n'est pas le cas avec Redis qui est un service tiers.

La prochaine interrogation est de savoir comment stocker l'état de la toile dans Redis. La première option est de stocker chaque pixel dans une clé Redis en utilisant les coordonnées comme clé. Cependant, récupérer un nombre important de clés en une fois (lors du chargement initial de la toile) n'est pas efficace. En effet, il faut scanner les différentes clés en spécifiant un paterne à trouver. Il est donc préférable de stocker l'état de la toile dans une seule clé Redis. Pour cela, il est possible d'utiliser le type de données Bitfield \cite{bitfield} de Redis. Ce type de données permet de stocker des bits dans une clé Redis. Il s'agit de la structure de données utilisée par l'équipe de Reddit dans leurs applications r/place de 2017 et 2022. Le Senior Software Engineer de Daniel Ellis a notamment donnée une conférence expliquant leur utilisation de Redis lors de la RedisConf 17 \cite{redisconf}. L'implémentation qui suit est donc basée sur les informations données dans cette conférence.

Afin de convertir la toile en une suite de bits, il est nécessaire de donner un id à chaque couleur afin de ne pas stocker un nombre incalculable de chaîne de caractères correspondant aux différentes couleurs. Pour avoir une marge concernant le nombre de couleurs disponibles, le type de chaque élément du Bitfield a été spécifié comme un entier non signé sur 8 bits. Cela nous permet de stocker 256 couleurs à la place des maximum 16 disponibles avec 4 bits. Le second avantage à utiliser 8 bits et qu'il rend possible l'utilisation d'un Uint8ClampedArray \cite{uint8clampedarray} du côté du client afin de créer une image du contenu du canvas très simplement.

Pour calculer l'offset du bit à modifier en fonction de la position du pixel, il est possible d'utiliser la simple formule suivante:
\begin{equation}
  offset = (y * width + x)
\end{equation}

L'avantage TODO

% Il est donc possible de stocker l'état de la toile dans une seule clé Redis en utilisant un bit par pixel. Cependant, il est nécessaire de calculer l'offset du bit à modifier en fonction de la position du pixel. Cela peut être fait en utilisant la formule suivante:

- Redis

Bitfield redis: https://redis.io/commands/bitfield/

Pour stocker tout le board d'un coup, on accède à une pos (x, y) en calculant l'offset

Autre option, hash redis avec comme clé (x, y): https://redis.io/docs/data-types/hashes/

\subsubsection{Stockage de l'historique}

% TODO
