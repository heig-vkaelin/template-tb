\section{Environnement existant}

Le code de tous les projets existants réalisés par l'équipe BeeScreen se trouvent dans un unique répertoire GitLab \cite{beescreens}. En effet, le répertoire est organisé sous la forme d'un monorepo. Ce qui signifie, qu'au lieu d'avoir une répertoire par projet comme il était surtout courant auparavant, tous les projets sont regroupés. Cette pratique a émergé au début des années 2000, notamment grâce aux grands noms de la tech comme Google, Meta ou encore Airbnb. Cela facilite notamment la collaboration entre les développeurs en ayant directement accès au code complet.

Le répertoire est organisé grâce à une workspace pnpm \cite{pnpmworkspace} afin de gérer le fait qu'il s'agit d'un monorepo. Un fichier de configuration est créé à la racines du répertoire en spécifiant la liste et le chemins vers tous les projets contenus dans le répertoire. Cela permettant notamment d'installer les dépendances de tous les projets en une seule commande.

pnpm est une version améliorée du gestionnaire de paquets par défaut présent avec Node.js npm. Il permet notamment de gérer les dépendances de manière plus efficace en utilisant des liens symboliques afin de ne pas avoir à copier les dépendances dans le dossier \mintinline[breaklines]{bash}{node_modules} de chaque projet. Cela permet de gagner en temps de compilation et en espace disque. De plus, comme évoqué précédemment, pnpm permet une gestion des monorepos plus efficace grâce à la notion de workspace.

\subsection{Structure du répertoire}

\dirtree{%
  .1 /.
  .2 apps.
  .3 media-player.
  .3 pimp-my-wall.
  .2 deployement.
  .3 apps.
  .4 ....
  .3 cicd.
  .3 infra.
  .2 docs.
  .2 packages.
  .3 pimp-my-wall.
}

Le dossier \mintinline[breaklines]{bash}{apps} contient les applications déjà développées qui sont pour l'instant:

\begin{enumerate}
  \item Media Player : une application permettant de lire des images ainsi que des vidéos afin d'afficher entre autre les sponsors et diverses informations lors du festival sous forme de diaporama.
  \item Pimp My Wall : l'application majeure du projet permettant aux utilisateurs de collaborer en temps réel en dessinant sur la même toile virtuelle. Cette application pourra servir de point de repère lors du développement de l'application BeePlace comme ces deux applications ont des fonctionnalités similaires (ex: temps réel, canvas partagé entre les utilisateurs, etc.).
\end{enumerate}

Le dossier \mintinline[breaklines]{bash}{deployement} contient les fichiers nécessaires au déploiement des diverses applications à l'aide de Docker ainsi que de Docker Compose. Chaque application possède son propre dossier à l'intérieur du dossier \mintinline[breaklines]{bash}{deployment/apps}. Les dossiers \mintinline[breaklines]{bash}{cicd} et \mintinline[breaklines]{bash}{infra} contiennent surtout des instructions sur la configuration des Raspberry Pis, utilisés afin de projeter les créations des utilisateurs de Pimp My Wall sur les murs du festival.

Le dossier \mintinline[breaklines]{bash}{docs} contient le code source de la documentation en ligne du projet \cite{beescreensdocs}. Celle-ci contient notamment divers tutoriels afin de créer de zéro des versions simplifiées des applications développées par l'équipe BeeScreens. De plus, divers guides ont également été créés afin de faciliter la prise en main des technologies utilisées dans le répertoire.

Pour finir, dossier \mintinline[breaklines]{bash}{packages} contient les différents paquets \Gls{npm} utilisés pour partager du code entre différentes applications (ex: entre la partie client et serveur d'une application).

% \subsection{CI/CD}

% TODO: CI/CD GitLab, déploiement

% \subsection{Tests}

% TODO: tests, husky

\subsection{Lien avec le projet BeePlace}
\label{sec:lien-avec-le-projet-beeplace}

BeePlace est une nouvelle application indépendante des applications existantes du projet BeeScreens. Cependant, comme expliqué précédemment, BeeScreens étant un monorepo, l'application BeePlace doit être ajoutée au même répertoire.

L'idée de BeePlace est arrivée suite aux conclusions réalisées à propos de l'application existante, Pimp My Wall. La problématique est la suivante: Pimp My Wall est une application trop permissive. Les utilisateurs peuvent dessiner de manière entièrement libre sur la toile virtuelle et cela donne suite à des comportements inappropriés. Le public cible, les festivaliers de Baleinev, est relativement jeune et l'ambiance festive de la soirée amène plus facilement à des dérives. Comme les créations des festivaliers sont affichés en temps réel sur les murs de l'école, il est nécessaire de modérer les créations des utilisateurs. Cette modération est fastidieuse, peu gratifiante et demande beaucoup de temps.

Afin de pallier à ce problème, l'idée de BeePlace est née. En imposant un délai entre la pose de pixels aux utilisateurs, ceux-ci seront moins enclins à réaliser des créations inadaptées car cela leur demanderait un effort plus conséquent. Dans le cas où des débordements auraient quand même lieu, la modération pourrait remettre à zéro une zone de la toile en quelques clics. Ce qui ferait perdre énormément de temps aux usagers problématiques et les dissuaderait de recommencer.

\section{Intégration}

Pour réaliser l'intégration de façon optimale, un guide \cite{addapptobeescreens} a été créé par les mainteneurs du projet BeeScreens expliquant les étapes à suivre afin d'ajouter une nouvelle application au répertoire.

\subsection{Création d'une nouvelle application}

La première étape consiste à créer le dossier de l'application dans le dossier \mintinline[breaklines]{bash}{apps} décrit précédemment. Dans celui-ci, deux dossiers sont également créés: \mintinline[breaklines]{bash}{frontend} et \mintinline[breaklines]{bash}{backend}. Le premier contient le code source de l'application côté client tandis que le second contient le code source de l'application côté serveur. D'éventuels autres dossiers seront peut être nécessaires en fonction des besoins de l'application. Ceux-ci seront créés après l'analyse préliminaire permettant de choisir les technologies utilisées.

L'arborescence est donc actuellement la suivante:

\dirtree{%
  .1 /.
  .2 apps.
  .3 beeplace.
  .4 backend.
  .4 frontend.
  .3 ....
  .2 ....
}

Par la suite, un fichier \mintinline[breaklines]{bash}{package.json} est nécessaire par dossier. Ces fichiers sont normalement utilisés dans les projets JavaScript et Node.js afin de préciser notamment les scripts, les dépendances et diverses métadonnées du projet. Dans ce cas-ci, ils sont utilisés peu importe les technologies qui seront choisies pour ce projet. En effet, cela permet d'unifier beaucoup d'aspects sur l'entièreté du code BeeScreens. Par exemple, il est possible de lancer les tests unitaires de toutes les applications grâce au fichier \mintinline[breaklines]{bash}{package.json} créé à la racine du répertoire. Si la future application n'est pas développée en JavaScript, les scripts du \mintinline[breaklines]{bash}{package.json} lancerons les commandes nécessaires dans le langage souhaité afin d'abstraire le plus possible les technologies utilisées.

La structure des fichiers \mintinline[breaklines]{bash}{package.json} est basée sur celle suggérée par les mainteneurs BeeScreens \cite{aboutpnpmbeescreens}. De nombreuses clés sont omises pour l'instant comme les technologies ne sont pas encore définies. La structure actuelle est la suivante:

\begin{listing}[h]
  \inputminted{json}{assets/figures/package.json}
  \caption{package.json initial du Backend l'application BeePlace}
\end{listing}

Celui du frontend suit la même structure. Un script de test a été ajouté afin de vérifier que l'intégration est réussie par la suite.

Un dossier ainsi qu'un fichier \mintinline[breaklines]{bash}{package.json} peuvent être créés dans le dossier \mintinline[breaklines]{bash}{packages} si du code doit être partagé entre les diverses parties de l'application. Ce besoin n'a pas encore été identifié donc aucun package n'a pour le moment été créé.

\subsection{Ajouter l'application au monorepo}

Afin de prévenir pnpm qu'une nouvelle application a été ajoutée à la workspace, il faut modifier le fichier \mintinline[breaklines]{bash}{pnpm-workspace.yaml} présent à la racine du répertoire en ajoutant les chemins vers les deux applications créées précédemment.


Grâce à cela, il est possible de lancer les tests précédemment créés en filtrant sur les applications souhaitées afin d'observer les résultats affichés dans le terminal.

\begin{minted}{bash}
  pnpm \
  --filter @beescreens/beeplace-backend \
  --filter @beescreens/beeplace-frontend \
  test

  # Résultat
  Scope: 2 of 11 workspace projects
  apps/beeplace/frontend test$ echo "This is a test from BeePlace Frontend!"
  │ This is a test from BeePlace Frontend!
  └─ Done in 9ms
  apps/beeplace/backend test$ echo "This is a test from BeePlace Backend!"
  │ This is a test from BeePlace Backend!
  └─ Done in 9ms
\end{minted}

Afin d'être sûr que l'IDE VSCode ait connaissance de la nouvelle application, il est également nécessaire d'ajouter les chemins dans le fichier de configuration \mintinline[breaklines]{bash}{.vscode/settings.json} présent à la racine du répertoire.

\subsection{CI/CD}

L'application doit être intégrées à deux pipelines: la locale gérée via Husky et la pipeline GitLab. Pour la première, il faut ajouter au fichier \mintinline[breaklines]{bash}{.husky/pre-push} le code suivant afin de lancer les tests avant chaque push:

\begin{minted}{bash}
## BeePlace
if
  did_files_change_in_directory "apps/beeplace/**/*"
then
  ## test
  pnpm --parallel \
  --filter @beescreens/beeplace-backend \
  --filter @beescreens/beeplace-frontend \
  test
fi
\end{minted}

Grâce à la fonction \mintinline[breaklines]{bash}{did_files_change_in_directory}, les tests ne seront lancés que si des fichiers ont été modifiés dans le dossier de l'application.

Pour la pipeline GitLab, il faut tout d'abord créer un nouveau fichier \mintinline[breaklines]{bash}{.gitlab/ci_cd/beeplace.yml}. Celui-ci contient les différentes étapes de la pipeline. Pour l'instant, il n'y a que la première étape qui est nécessaire: le lancement des "tests".

\begin{minted}{yaml}
## Backend

# test
beeplace::app::backend::test:
  needs:
  - setup env
  extends: .test
  variables:
  PROJECT_PATH: "apps/beeplace/backend"
  PROJECT_NAME: "@beescreens/beeplace-backend"
  script:
  - echo "Testing $PROJECT_NAME..."

## Frontend

# test
beeplace::app::frontend::test:
  needs:
  - setup env
  extends: .test
  variables:
  PROJECT_PATH: "apps/beeplace/frontend"
  PROJECT_NAME: "@beescreens/beeplace-frontend"
  script:
  - echo "Testing $PROJECT_NAME..."
\end{minted}

Ce fichier doit être inclus dans le fichier \mintinline[breaklines]{bash}{.gitlab/ci_cd/main.yml} afin d'être ajouté à la pipeline globale.

L'application est maintenant prête à être développée. Le code peut être push afin de lancer les tests en local avec Husky et sur la pipeline GitLab.