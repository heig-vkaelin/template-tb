% Contexte
L'association Baleinev organise depuis près de 30 ans le festival de musique Baleinev Festival sur le campus de la HEIG-VD. Depuis 2014, le festival propose un concept innovant nommé Pimp My Wall, une application collaborative de dessins en temps réel sur les fenêtres des tours de l'école.

% Problématique
Le concept de BeePlace se base sur la conclusion que Pimp My Wall est une application trop permissive. Les utilisateurs peuvent dessiner de manière entièrement libre sur la toile virtuelle et cela engendre quelques comportements inappropriés.

% Réalisation
En recréant l'expérience proposée par le r/place de Reddit, l'objectif est de proposer une application web dans laquelle les utilisateurs peuvent créer des œuvres d'art collaboratives. Chaque utilisateur peut dessiner un pixel à la fois sur la toile partagée et doit attendre un certain temps avant de pouvoir dessiner à nouveau. Cela permet de forcer la collaboration et de limiter les débordements.

Pour les festivaliers, l'application a été pensée en premier lieu pour une utilisation mobile en faisant usage de technologies telles que Next.JS, NestJS et Redis, parmi d'autres.
Un des points principaux de ce travail, en plus de la réalisation de l'application, était d'assurer la montée en charge afin de garantir la fluidité de l'application en tout temps. Cela a pu être réalisé grâce à k6 et Clinic.js, aboutissant à un gain de 142\% de joueurs par rapport à la version initiale, pour un total estimé de 1350 joueurs en parallèle.

% Perspectives
Pour finir, cette application sera peaufinée et testée lors de la prochaine édition du Baleinev Festival, afin de valider sa pertinence pour les festivaliers et son comportement dans un contexte réel. L'application est d'ores et déjà accessible à l'adresse \href{https://place.beescreens.ch}{place.beescreens.ch}.
