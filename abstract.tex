% contexte
L'association Baleinev organise depuis près de 30 ans le festival de musique Baleinev Festival sur le campus de l'HEIG.
Depuis 2014, le festival propose un concept innovant appelé Pimp My Wall. L'idée consiste à utiliser les fenêtres de l'école comme des écrans géants pour pouvoir notamment dessiner en temps réel depuis un smartphone.

Le concept a été repris de zéro en 2018 et a donné naissance à BeeScreens, la nouvelle version open source aux technologies modernes. Les ambitions ont été revues à la hausse avec l'idée de proposer de nombreuses applications interactives diffusées en continu sur Internet.
C'est dans ce contexte-ci qu'est né le projet de ce travail de Bachelor qui cherche à créer une
nouvelle application interactive à projeter entre autres sur un sous-ensemble des écrans du festival.

\asterism

% problématique
Le concept de BeePlace se base sur la conclusion que Pimp My Wall est une application trop permissive. Les utilisateurs peuvent dessiner de manière entièrement libre sur la toile virtuelle et cela engendre des comportements inappropriés.

% objectifs du travail
En recréant l'expérience proposée par le r/place de Reddit, l'objectif est de proposer une application web dans laquelle les utilisateurs peuvent créer des \oe{}uvres d'art collaboratives. Chaque utilisateur peut dessiner un pixel à la fois sur la toile et doit attendre un certain temps avant de pouvoir dessiner à nouveau. Cela permet de forcer la collaboration et de limiter les débordements.

Un des points principaux de ce travail en plus de la réalisation de l'application est d'assurer la montée en charge. L'application doit rester fluide dans le cas où de nombreuses personnes présentes au festival la rejoignent. Il est donc nécessaire de mettre en place des tests de montée en charge afin de pouvoir déterminer les limites de l'application et d'optimiser le code afin d'avoir des résultats satisfaisants.

% perspectives
Pour finir, le code de l'application a été pensé pour pouvoir être modifié, amélioré et étendu facilement pour s'adapter aux besoins futurs. Il ne s'agit pas d'un produit fini mais d'une base de travail pour les années à venir.
